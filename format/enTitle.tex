% !TeX root=main.tex
% در این فایل، عنوان پایان‌نامه، مشخصات خود و چکیده پایان‌نامه را به انگلیسی، وارد کنید.

%%%%%%%%%%%%%%%%%%%%%%%%%%%%%%%%%%%%

\begin{latin}
  %\latinfaculty{}
  \latindepartment{School of Electrical and Computer Engineering}
  \latinsubject{Electrical Engineering}
  \latinfield{Communication Engineering}
  \latintitle{Virtualized Internet of Things Platform for Smart City Applications}
  \firstlatinsupervisor{Dr. Vahid Shah-Mansouri}
  %\secondlatinsupervisor{Second Supervisor}
  %\firstlatinadvisor{First Advisor}
  %\secondlatinadvisor{Second Advisor}
  \latinname{Ali}
  \latinsurname{Sorour Amini}
  \latinthesisdate{September 2019}
  
  \latinkeywords{Internet of Things (IoT), Smart City, edge computing, resource assignment}
  
  \en-abstract{
    By moving towards the Internet of Things (IoT) era, the number of connected devices to the internet is increasing exponentially.
    Such a large number of devices creates bottlenecks in several areas including device connectivity, data transport, and data processing.
    Edge computing, a computation paradigm trying to process data at the edge of the network, is a promising solution for the processing of the large amount of data produced by this vast number of connected devices.
    Thanks to new advances in the container based virtualization, it is now possible for network gateways and current edge devices to share their extra computation capacity with IoT services for the purpose of data processing.
    Processing tasks of IoT services can be assigned to the edge devices to offload traffic towards the cloud and to reduce the service delay.
    The vast number of these edge devices and services in IoT networks makes the assignment of computing resources to the services a challenging problem.
    In this paper, we mathematically formulate the problem of assignment of computation resources (i.e. edge or cloud) to IoT services in the IoT network.
    The problem is modeled as an optimization problem with the objective of maximizing the total utility of the services.
    The resulting optimization problem is a mixed integer non-linear program which is generally hard to solve.
    Proposed algorithms result in near optimal solutions and well suited for distributed and asynchronous deployment.
  }

  \cleartoleftpage
  \latinabstractPage
  \cleartoleftpage
  \latinfirstPage
\end{latin}
